\documentclass{article}
\usepackage{graphicx}
\usepackage{amsmath}
\usepackage[T1]{fontenc}
\usepackage{lmodern}
\usepackage{enumitem}
\usepackage{hyperref}

\setlength{\parindent}{0pt}
\hypersetup{
  colorlinks, linkcolor=red
}

\begin{document}

``And where a mathematical reasoning can be had, it is as great a folly to make
use of any other, as to grope for a thing in the dark, when you have a candle
standing by you'' -- John Arbuthnot, \textit{On The Laws of Chance}
\newline

\rule{12cm}{0.4pt}

\par 
Notes on notation:

$\in$ \quad Element of
\newline
$\subseteq$ \quad Subset of
\newline
$\subset$ \quad Proper subset of
\newline
$\supseteq$ \quad Superset of
\newline
$\supseteq$ \quad Proper superset of
\newline
$\theta$ \quad Empty set
\newline
$\cup$ \quad Set union
\newline
$\cap$ \quad Set intersection
\newline
$\times$ \quad Cartesian product
\newline
$2^A$ \quad Powerset of set $A$
\newline
$|\ |$ \quad Set cardinality
\newline
$\{\ \}$ \quad Set
\newline
$<\ >$ \quad Ordered tuple
\newline
$(\ )$ \quad Generator
\newline
$[\ ]$ \quad Class generated by a generator
\newline
$(A - B)$ \quad ``set difference'' between A and B (or
``relative complement of A with respect to B''). IOW, ``the set of elements in A
but not in B'' See:\newline
\url{https://en.wikipedia.org/wiki/Complement_(set_theory)}\newline
\rule{12cm}{0.4pt}
\newline

\par
\textbf{Definition} 3.1: A \textit{diagnostic problem} P is a 4-tuple \textless
D, M, C, M$^{+}$\textgreater {} where $D = \{d_1, d_2, \ldots, d_n\}$ is a finite,
non-empty set of objects, called disorders, $M = \{m_1, m_2, \ldots, m_n\}$ is a
finite, non-empty set of objects called manifestations, and $C \subseteq D
\times M$ is a relation with \textit{domain}(C) = D and \textit{range}(C) = M,
called causation, and $M^{+} \subseteq M$ is a distinguished subset of M which is said to be
\textbf{present}.
\newline

\par 
\textbf{Definition} 3.2: For any element $d_i \in D$ and $m_j \in M$ in a
diagnostic problem \textless D, M, C, M$^{+}$\textgreater,
\textit{effects}($d_i) = \{ m_j\:|\:\langle d_i, m_j\rangle \in C\}$, the set
of objects directly caused by $d_i$, and 
\textit{causes}($m_j) = \{ d_i\:|\:\langle d_i, m_j\rangle \in C\}$, the set of
objects which can directly cause $m_j$
\newline


\par 
\textbf{Definition} 3.3: For any $D_I \subseteq D$ and $M_J \subseteq M$ in a
diagnostic problem \textless D, M, C, M$^{+}$\textgreater,
\textit{effects}($D_I$) = $\bigcup\limits_{d_i \in D_I} effects(d_i)$, and
\textit{causes}($M_J$) = $\bigcup\limits_{m_j \in M_J} causes(m_j)$
\newline
\par 
Thus, for example, the effects of a set of disorders are just the union
(``sum'') of effects of individual disorders in the set.
\newline


\textbf{Definition of \textit{COVER}}
\newline
\par 
\textbf{Definition} 3.4: The set $D_I \subseteq D$ is said to be a
\textit{cover} of $M_J \subseteq M$ if $M_J \subseteq effects(D_I)$
\newline
In other words, $M_J$ is \textit{covered} by $D_I$ if every manifestation in
$M_J$ is causally associated with some member(s) of $D_I$.  Or, to put it
another way, the set of disorders $D_I$ is sufficient to explain every 
manifestation in $M_J$.


\par 
\textbf{Definition} 3.5: A set $E \subseteq D$ is said to be an
\textit{explanation} of M$^{+}$ for a problem P = \textless
D, M, C, M$^{+}$\textgreater if E \textit{covers} M$^{+}$ and E satisfies a
given parsimony condition.
\newline

\textbf{Definition of \textit{IRREDUNDANT}}
\newline
\par 
\textbf{Definition} 3.6: 
\begin{enumerate}
\item A cover, $D_I$ of $M_J$ is said to be \textit{minimum}
if its cardinality is smallest among all covers of $M_J$.
\item A cover, $D_I$ of $M_J$ is said to be \textit{irredundant} if none of its
proper subsets is also a cover of $M_J$. It is said to be \textit{redundant}
otherwise.
\item A cover, $D_I$ of M$^{+}$ is said to be \textit{relevant} if it is a
subset of $causes(M^{+})$; it is \textit{irrelevant} otherwise.
\end{enumerate}
\par
In other words, a cover is irredundant if you can't take away any of its
members and have it remain a cover.  There may be covers with smaller
cardinality though, because they may be made up of disorders that - individually
- explain more manifestations.  
\newline
Similarly, a cover is relevant if all of its members are a cause of at least
one of the manifestations. If a cover contains any disorders that explain none
of the manifestations that are present, then it is called \textit{irrelevant}.
\newline


\par 
\textbf{Definition} 3.7: The \textit{solution} to a diagnostic problem
P = \textless D, M, C, M$^{+}$\textgreater designated $Sol(P)$ is the set of all
explanations of M$^{+}$.
\newline

\par 
\textbf{Lemma} 3.1: Let P = \textless D, M, C, M$^{+}$\textgreater be the causal
network for a diagnostic problem and $d_i \in D, m_j \in M, D_I, D_K \subseteq
D, \text{and} M_J \subseteq M$, then:
\begin{enumerate}[label={(\alph*)}]
  \item $effects(d_i) \neq \theta\text{,}\  causes(m_j) \neq \theta$
  \item $d_i \in causes(effects(d_i))\text{,}\ m_j \in effects(causes(m_j))$
  \item $D_I \subseteq causes(effects(D_I))\text{,}\ M_J \subseteq
  effects(causes(M_J))$
  \item $M = effects(D)\text{,}\ D = causes(M)$
  \item $d_i \in causes(m_j)\  \text{iff}\  m_j \in effects(d_i)$
  \item $effects(D_I) - effects(D_K) \subseteq effects(D_I - D_K)$
\end{enumerate}

\par 
\textbf{Lemma} 3.2: If P = \textless D, M, C, M$^{+}$\textgreater is the causal
network for a diagnostic problem with $D_I \subseteq D$ and $M_J \subseteq M$,
then $D_I \cap causes(M_J) = \theta\ \text{iff}\ M_J \cap effects(D_I) =
\theta$.
\newline

\par 
\textbf{Lemma} 3.3: If $D_K$ is a cover of $M_J$ in a diagnostic problem, then
there exists a $D_I \subseteq D_K$ which is an irredundant cover of $M_J$.
\newline

\par 
\textbf{Thereom} 3.4: (Explanation Existence Theorem) There exists at least one
explanation for M$^{+}$ for any diagnostic problem P = \textless D, M, C,
M$^{+}$\textgreater
\newline
Follows from Lemma 3.1 (d) and Lemma 3.3
\newline

\par 
\textbf{Lemma} 3.5: A cover $D_I$ of $M_J$ is irredundant iff for every $d_i
\in D_I$ there exists some $m_j \in M_J$ which is uniquely covered by $d_i$,
i.e., $m_j \in effects(d_i)$ but $m_j \not\in effects(D_I - \{d_i\} )$
\newline

\par 
\textbf{Lemma} 3.6: If $D_I$ is an irredundant cover of $M_J$ then $|D_I| \leq
|M_J|$. More specifically, if E is an explanation of M$^{+}$ for a diagnostic
problem, then $|E| \leq |M^{+}|$.
\newline

\par 
\textbf{Lemma} 3.7: $E = \theta$ is the only explanation for $M^{+} = \theta$.
\newline

\par 
\textbf{Thereom} 3.8: (Competing Disorders Theorem) Let E be an explanation for
M$^{+}$, and let $M^{+} \cap effects(d_1) \subseteq M^{+} \cap effects(d_2)$ for
some $d_1, d_2 \in D$. Then, 
\begin{enumerate}
  \item $d_1\ \text{and}\ d_2$ are not both in E; and
  \item if $d_1 \in E$, then there is another explanation $E'$ for M$^{+}$
  containing $d_2$ but not $d_1$, of equal or smaller cardinality.
\end{enumerate}

\par 
\textbf{Lemma} 3.9: Let $2^D$ be the power set of $D$, and let $S_{mc},\
S_{ic},\ S_{rc},\ \text{and}\ S_c$ be sets of all minimum covers, all
irredundant covers, all redundant covers, and all covers of M$^{+}$
respectively, for a diagnostic problem. Then $\theta \subseteq S_{mc}
\subseteq S_{ic} \subseteq S_{rc} \subseteq S_c \subseteq 2^D$.
\newline

\par 
\textbf{Lemma} 3.10: (Subsumption Property) - For a diagnostic problem P, let
$S_{ic}$ be the set of all irredundant covers of M$^{+}$. Then $S_{ic}$ is the
smallest set of covers such that for any $D_K \subseteq D$ covering M$^{+}$,
there is a $D_I$ in that set of covers with $D_I \subseteq D_K$.
\newline



\par 
\textbf{Definition} 3.8: Let $g_1, g_2, \ldots, g_n$ be non-empty,
pairwise-disjoint subsets of $D$. Then $G_I = \{ g_1, g_2, \ldots, g_n\}$ is a
\textit{generator}.  The class generated by $G_I$, designated as $[G_I]$ is
defined to be $[G_I] = \{ \{d_1, d_2, \ldots, d_n  \} \:|\: d_i \in g_i, 1 \leq
i \leq n \}$.
\newline
Note: what does ``pairwise disjoint'' mean?  See: \url{https://en.wikipedia.org/wiki/Disjoint_sets}
\newline
In mathematics, two sets are said to be disjoint sets if they have no element in common. Equivalently,
disjoint sets are sets whose intersection is the empty set.
For example, {1, 2, 3} and {4, 5, 6} are disjoint sets, while {1, 2, 3} and {3, 4, 5} are not.
\newline
This definition of disjoint sets can be extended to any family of sets. A family of sets is pairwise disjoint
or mutually disjoint if every two different sets in the family are disjoint.
For example, the collection of sets { {1}, {2}, {3}, ... } is pairwise disjoint.
\newline
Note: What do we mean by ``class''? See: \url{https://en.wikipedia.org/wiki/Class_(set_theory)}
\newline
In set theory and its applications throughout mathematics, a class is a collection of sets (or sometimes
other mathematical objects) that can be unambiguously defined by a property that all its members share.
The precise definition of "class" depends on foundational context.
\newline
Examples:\newline
The collection of all algebraic objects of a given type will usually be a proper class. Examples
include the class of all groups, the class of all vector spaces, and many others.
\newline
The surreal numbers are a proper class of objects that have the properties of a field.
\newline
Within set theory, many collections of sets turn out to be proper classes. Examples include the
class of all sets, the class of all ordinal numbers, and the class of all cardinal numbers.
\newline
Note: Definition of ``Generator'' in mathematics:
\newline
\url{https://en.wikipedia.org/wiki/Generator_(mathematics)}
\newline
\par
Regarding the pairwise disjoint subsets that go into a generator:  The members
of any one such set are \textit{competing possibilities} where one such
possibility, combined with one possibility each from the other sets, makes up
one discrete hypothesis.  To put it another way, within each
generator-member-set, (or competing-possibility-set) the members compete to
explain some portion of the present manifestations in $M^{+}$.
\newline

\par 
\textbf{Definition} 3.9: $G = \{G_1, G_2, \ldots, G_N\}$ is a
\textit{generator-set} if each $G_I \in G$ is a generator and $[G_I] \cap
[G_J] = \theta\ \text{for}\ I \neq J$. 
The class generated by G is $[G] = \bigcup\limits_{I=1}^{N}[G_I]$.
\newline
\newline
\par
\textit{\textbf{DIVISION}}
\newline
\par
Using the disorders evoked by a newly discovered manifestation, a division
operation selects from existing hypotheses those which cover both the old
manifestations AND the new manifestation.
\newline
\par
\textbf{Definition} 3.10: Let $G_I = (g_1, g_2, \ldots, g_n)$ be a generator 
and let $H_1 \subseteq D$ where $H_1 \neq \theta$. 

Then $Q = \{ Q_k \:|\:Q_k$ is a generator $\}$ is a division of $G_I$ by $H_1$
if for all k, $1 < k < n$, $Q_k$ = $( \, q_{k1}, q_{k2}, \ldots, q_{kn} ) \,$ where
\[
  q_{kj} =
	\begin{cases}
		g_j - H_1, \text{ if } j < k,\\
		g_j \cap H_1, \text{ if } j = k,\\
		g_j, \text{ if } j > k
	\end{cases}
\]
\par
Informally, all generators resulting from a division can be considered to be
calculated as follows: \newline
For any given order of $g_j$'s in $G_I$, the first generator in the division of
$G_I$ by $H_1$ is the same as $G_I$ except $g_1$ is replaced by $g_1 \cap H_1$.
In the second generator, $g_1$ is replaced by $g_i - H_1$, $g_2$ is replaced by
$g_2 \cap H_1$, other $g_j$'s are not changed, and so on. In the $K^{th}$
generator, all $g_j$'s prior to $g_k$ are replaced by $g_j - H_1$, $g_k$ is
replaced by $g_k \cap H_1$, and all $g_j$'s after $g_k$ are unchanged. 
\par
It is clear from the foregoing definition that for any generator $Q_k$ resulting
from division of $G_I$ by $H_1$, one set it contains - namely $q_{kk}$, is a
subset of $H_1$. The set difference operations for $j < k$ are to make sure that
the classes for different generators resulting from the division are disjoint
and thus to ensure that $Q$ is a generator-set. 
\newline
\par 
\textbf{Definition} 3.11: Let G be a generator-set and $H_1 \subseteq D\
\text{where}\ H_1 \neq \theta$. A divsion of G by $H_1$ is $div(G, H_1) =
\bigcup\limits_{G_I \in G} div(G_I, H_1)$
\newline

\par 
\textbf{Lemma} 3.11: Let $G_I$ be a generator, $G$ be a generator-set, and $H_1
\subseteq D$ where $H_1 \neq \theta$. Then:
\begin{enumerate}[label={(\alph*)}]
  \item $div(G_I, H_1)$ is a generator-set with $[div(G_I, H_1)] = \{ E \in
  [G_I] \:|\: E \cap H_1 \neq \theta \}$; and
  \item $div(G, H_1)$ is a generator-set with $[div(G, H_1)] = \{ E \in [G]
  \:|\: E \cap H_1 \neq \theta \}]$\newline
\end{enumerate}

\par 
\textbf{Definition} 3.12: Let $G_I = (g_1, g_2, \ldots, g_n)$ be a generator,
$G$ a generator-set, and $H_1 \subseteq D$ where $H_1 \neq \theta$. Then the
\textit{residual of division} of $G_I$ by $H_1$ is 
\[
res(G_I, H_1) =
\begin{cases}
   \{ (g_1 - H_1, \ldots, g_n - H_1) \}, \text{ if } g_i - H_1 \neq \theta
   \text{ for all } i, 1 \leq i \leq n
   \\
   \theta, \text{ otherwise}
\end{cases}
\] 
and the \textit{residual of division} of $G$ by $H_1$ is \newline
$res(G, H_1) = \bigcup\limits_{G_I \in G} res(G_I, H_1)$
\newline

\par 
\textbf{Lemma} 3.12: For $G_I, G, \text{ and } H_1$, as defined in Definition
3.12, 
\begin{enumerate}[label={(\alph*)}]
  \item $res(G_I, H_1)$ is a generator set with $[res(G_I, H_1)] = \{ E \in
  [G_I] \:|\: E \cap H_1 = \theta \}$; and
  \item $res(G, H_1)$ is a generator set with $[res(G, H_1)] = \{ E \in [G]
  \:|\: E \cap H_1 = \theta \}$\newline
\end{enumerate}

\par 
\textbf{Definition} 3.13: Let G and Q be generator-sets, $G_I \in G$ and $Q_J
\in Q$ be generators, and $q_j \in Q_J$. Then a division of $G_I$ by $Q_J$ is
\[
   div( G_I, Q_J) = 
   \begin{cases}
   \{G_I\}, \text{ if } Q_J = \theta \\
   div( div( G_I, q_j), Q_I - (q_j)), \text{ otherwise}
   \end{cases}
\]
A divsion of G by $Q_J$ is $div(G, Q_J) = \bigcup\limits_{G_I \in G} div( G_I,
Q_J)$ \newline+

\par 
\textbf{Lemma} 3.13: Let G, $G_I$, $Q_J$ and $q_j$ be as defined in Definition
3.13.  Then:
\begin{enumerate}[label={(\alph*)}]
  \item $div(G_I, Q_J)$ is a generator-set with $[div(G_I, Q_J)] = \{ E \in
  [G_I]\;|\; \text{ there exists } E' \in [Q_J] \text{ where } E' \subseteq E
  \}$
  \item $div(G, Q_J)$ is a generator set with $[div(G, Q_J)] = \{ E \in [G]
  \;|\; \text{ there exists } E' \in [Q_J] \text{ where } E' \subseteq E \}$
\end{enumerate}

\par 
\textbf{Definition} 3.14: Let G and Q be generator-sets, $G_I \in G$ and $Q_J
\in Q$ be generators, $q_j \in Q_J$. Then a \textit{ residual of division} of
$G_I$ by $Q_J$ is
\[
 res(G_I, Q_J) = \begin{cases}
 \theta \text{ if } Q_J = \theta \\
 res(G_I, q_j) \cup res( div( G_I, q_j), Q_J - (q_j)), \text{ otherwise}
 \end{cases}
\]
\newline
A \textit{residual of division} of $G$ by $Q_J$ is
\[
    res( G, Q_J) = \bigcup\limits_{G_I \in G} res(G_I, Q_J)\\
\]
\newline
And a \textit{residual of division} of $G$ by $Q$ is
\[
res( G, Q) = \begin{cases}
  G, \text{ if } Q= \theta, \\
  res( res( G, Q_J), Q- \{Q_J\}), \text{ otherwise}
\end{cases}
\]
\newline

\par 
\textbf{Lemma} 3.14: Let $G$, $Q$, $G_I$, $Q_J$, and $q_j$ be as defined in
definition 3.14. Then:
\begin{enumerate}[label={(\alph*)}]
  \item $res( G_I, Q_J)$ is a generator-set with \newline $[res(G_I, Q_J)] = \{
   E \in [G_J] \:|\: \text{ there does not exist } E' \in [Q_J] \text{ where }
   E' \subseteq E \}$
  \item $res( G, Q_J)$ is a generator-set with $[res(G, Q_J)] = \{ E \in [G]
  \:|\: \text{ there does not exist } E' \in [Q_J] \text{ where } E' \subseteq E
  \}$
  \item $res( G, Q)$ is a generator-set with: $[res(G, Q)] = \{ E \in [G] \::\:
  \text{ there does not exist } E' \text{ where } E' \subseteq E \}$
\end{enumerate}


\par 
\textbf{Definition} 3.15: Let $G_I = ( g_1, g_2, \ldots, g_n)$ be a
generator, $G$ a generator-set, and $H_1 \subseteq D$ where $H_1 \neq \theta$.
Then the \textit{augmented residual} of division of $G_I$ by $H_1$ is
\[
augres(G_I, H_1) = 
\begin{cases}
 \{ (g_1 - H1, \ldots, g_n - H_1, A) \} \text{ if } g_i - H_1 \neq \theta, I
 \leq i \leq n, A \neq \theta
 \\
 \theta, \text{ otherwise}
\end{cases}
\]
\newline
$\text{ where } A = H_1 - \bigcup\limits_{i=1}^{n} g_i
\newline
\text{ The augmented residual of division of G by } H_1 \text{ is }$
\[
augres(G, H_1) = \bigcup\limits_{G_I \in G}augres(G_I, H_1)
\]
\newline

\par 
\textbf{Lemma} 3.15: Let $G_I$, $G$, and $H_1$ be as defined in Definition 3.15.
Then $augres( G_I, H_1) \text{ and } augres(G, H_1)$ are generator-sets.
\newline

----
\newline
Algorithm BIPARTITE
\newline



----

\par 
\textbf{Lemma} 3.16:
\newline

\par 
\textbf{Lemma} 3.17:
\newline

\par 
\textbf{Lemma} 3.18:
\newline

\par 
\textbf{Thereom} 3.19:
\newline


\par 
\textbf{Definition} 3.16:
\newline

\par 
\textbf{Lemma} 3.20:
\newline

\par 
\textbf{Definition} 3.17:
\newline

\par 
\textbf{Lemma} 3.21:
\newline


\par 
\textbf{Definition} 3.18:
\newline

\par 
\textbf{Lemma} 3.22:
\newline

\par 
\textbf{Thereom} 3.23:
\newline

\par 
\textbf{Definition} 3.19:
\newline

\par 
\textbf{Definition} 3.20:
\newline

\par 
\textbf{Lemma} 3.24:
\newline

TBD

\par 
\textbf{Thereom} 3.25:
\newline










\end{document}
