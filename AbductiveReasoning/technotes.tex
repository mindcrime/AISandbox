\documentclass{article}
\usepackage{graphicx}
\usepackage{amsmath}
\usepackage[T1]{fontenc}
\usepackage{lmodern}
\usepackage{enumitem}
\usepackage{hyperref}

\setlength{\parindent}{0pt}
\hypersetup{
  colorlinks, linkcolor=red
}

\begin{document}

``And where a mathematical reasoning can be had, it is as great a folly to make
use of any other, as to grope for a thing in the dark, when you have a candle
standing by you'' -- John Arbuthnot, \textit{On The Laws of Chance}
\newline

\rule{12cm}{0.4pt}

\par 
Notes on notation:

$\in$ \quad Element of
\newline
$\subseteq$ \quad Subset of
\newline
$\subset$ \quad Proper subset of
\newline
$\supseteq$ \quad Superset of
\newline
$\supseteq$ \quad Proper superset of
\newline
$\theta$ \quad Empty set
\newline
$\cup$ \quad Set union
\newline
$\cap$ \quad Set intersection
\newline
$\times$ \quad Cartesian product
\newline
$2^A$ \quad Powerset of set $A$
\newline
$|\ |$ \quad Set cardinality
\newline
$\{\ \}$ \quad Set
\newline
$<\ >$ \quad Ordered tuple
\newline
$(\ )$ \quad Generator
\newline
$[\ ]$ \quad Class generated by a generator
\newline
$(A - B)$ \quad ``set difference'' between A and B (or
``relative complement of A with respect to B''). IOW, ``the set of elements in A
but not in B'' See:\newline
\url{https://en.wikipedia.org/wiki/Complement_(set_theory)}\newline
\rule{12cm}{0.4pt}
\newline

\par
\textbf{Definition} 3.1: A \textit{diagnostic problem} P is a 4-tuple \textless
D, M, C, M$^{+}$\textgreater {} where $D = \{d_1, d_2, \ldots, d_n\}$ is a finite,
non-empty set of objects, called disorders, $M = \{m_1, m_2, \ldots, m_n\}$ is a
finite, non-empty set of objects called manifestations, and $C \subseteq D
\times M$ is a relation with \textit{domain}(C) = D and \textit{range}(C) = M,
called causation, and $M^{+} \subseteq M$ is a distinguished subset of M which is said to be
\textbf{present}.
\newline

\par 
\textbf{Definition} 3.2: For any element $d_i \in D$ and $m_j \in M$ in a
diagnostic problem \textless D, M, C, M$^{+}$\textgreater,
\textit{effects}($d_i) = \{ m_j\:|\:\langle d_i, m_j\rangle \in C\}$, the set
of objects directly caused by $d_i$, and 
\textit{causes}($m_j) = \{ d_i\:|\:\langle d_i, m_j\rangle \in C\}$, the set of
objects which can directly cause $m_j$
\newline

\par 
\textbf{Definition} 3.3: For any $D_I \subseteq D$ and $M_J \subseteq M$ in a
diagnostic problem \textless D, M, C, M$^{+}$\textgreater,
\textit{effects}($D_I$) = $\bigcup\limits_{d_i \in D_I} effects(d_i)$, and
\textit{causes}($M_J$) = $\bigcup\limits_{m_j \in M_J} causes(m_j)$
\newline
\par 
Thus, for example, the effects of a set of disorders are just the union
(``sum'') of effects of individual disorders in the set.
\newline

\par 
\textbf{Definition} 3.4: The set $D_I \subseteq D$ is said to be a
\textit{cover} of $M_J \subseteq M$ if $M_J \subseteq effects(D_I)$
\newline

\par 
\textbf{Definition} 3.5: A set $E \subseteq D$ is said to be an
\textit{explanation} of M$^{+}$ for a problem P = \textless
D, M, C, M$^{+}$\textgreater if E \textit{covers} M$^{+}$ and E satisfies a
given parsimony condition.
\newline

\par 
\textbf{Definition} 3.6: 
\begin{enumerate}
\item A cover, $D_I$ of $M_J$ is said to be \textit{minimum}
if its cardinality is smallest among all covers of $M_J$.
\item A cover, $D_I$ of $M_J$ is said to be \textit{irredundant} if none of its
proper subsets is also a cover of $M_J$. It is said to be \textit{redundant}
otherwise.
\item A cover, $D_I$ of M$^{+}$ is said to be \textit{relevant} if it is a
subset of $causes(M^{+})$; it is \textit{irrelevant} otherwise.
\end{enumerate}

\par 
\textbf{Definition} 3.7: The \textit{solution} to a diagnostic problem
P = \textless D, M, C, M$^{+}$\textgreater designated $Sol(P)$ is the set of all
explanations of M$^{+}$.
\newline

\par 
\textbf{Lemma} 3.1: Let P = \textless D, M, C, M$^{+}$\textgreater be the causal
network for a diagnostic problem and $d_i \in D, m_j \in M, D_I, D_K \subseteq
D, \text{and} M_J \subseteq M$, then:
\begin{enumerate}[label={(\alph*)}]
  \item $effects(d_i) \neq \theta\text{,}\  causes(m_j) \neq \theta$
  \item $d_i \in causes(effects(d_i))\text{,}\ m_j \in effects(causes(m_j))$
  \item $D_I \subseteq causes(effects(D_I))\text{,}\ M_J \subseteq
  effects(causes(M_J))$
  \item $M = effects(D)\text{,}\ D = causes(M)$
  \item $d_i \in causes(m_j)\  \text{iff}\  m_j \in effects(d_i)$
  \item $effects(D_I) - effects(D_K) \subseteq effects(D_I - D_K)$
\end{enumerate}

\par 
\textbf{Lemma} 3.2: If P = \textless D, M, C, M$^{+}$\textgreater is the causal
network for a diagnostic problem with $D_I \subseteq D$ and $M_J \subseteq M$,
then $D_I \cap causes(M_J) = \theta\ \text{iff}\ M_J \cap effects(D_I) =
\theta$.
\newline

\par 
\textbf{Lemma} 3.3: If $D_K$ is a cover of $M_J$ in a diagnostic problem, then
there exists a $D_I \subseteq D_K$ which is an irredundant cover of $M_J$.
\newline

\par 
\textbf{Thereom} 3.4: (Explanation Existence Theorem) There exists at least one
explanation for M$^{+}$ for any diagnostic problem P = \textless D, M, C,
M$^{+}$\textgreater
\newline
Follows from Lemma 3.1 (d) and Lemma 3.3
\newline

\par 
\textbf{Lemma} 3.5: A cover $D_I$ of $M_J$ is irredundant iff for every $d_i
\in D_I$ there exists some $m_j \in M_J$ which is uniquely covered by $d_i$,
i.e., $m_j \in effects(d_i)$ but $m_j \not\in effects(D_I - \{d_i\} )$
\newline

\par 
\textbf{Lemma} 3.6: If $D_I$ is an irredundant cover of $M_J$ then $|D_I| \leq
|M_J|$. More specifically, if E is an explanation of M$^{+}$ for a diagnostic
problem, then $|E| \leq |M^{+}|$.
\newline

\par 
\textbf{Lemma} 3.7: $E = \theta$ is the only explanation for $M^{+} = \theta$.
\newline

\par 
\textbf{Thereom} 3.8: (Competing Disorders Theorem) Let E be an explanation for
M$^{+}$, and let $M^{+} \cap effects(d_1) \subseteq M^{+} \cap effects(d_2)$ for
some $d_1, d_2 \in D$. Then, 
\begin{enumerate}
  \item $d_1\ \text{and}\ d_2$ are not both in E; and
  \item if $d_1 \in E$, then there is another explanation $E'$ for M$^{+}$
  containing $d_2$ but not $d_1$, of equal or smaller cardinality.
\end{enumerate}

\par 
\textbf{Lemma} 3.9: Let $2^D$ be the power set of $D$, and let $S_{mc},\
S_{ic},\ S_{rc},\ \text{and}\ S_c$ be sets of all minimum covers, all
irredundant covers, all redundant covers, and all covers of M$^{+}$
respectively, for a diagnostic problem. Then $\theta \subseteq S_{mc}
\subseteq S_{ic} \subseteq S_{rc} \subseteq S_c \subseteq 2^D$.
\newline

\par 
\textbf{Lemma} 3.10: TBD
\newline



\par 
\textbf{Definition} 3.8: TBD
\newline


\par 
\textbf{Definition} 3.9: TBD
\newline


\par 
\textbf{Definition} 3.10: Let $G_I = (g_1, g_2, \ldots, g_n)$ be a generator 
and let $H_1 \subseteq D$ where $H_1 \neq \theta$. 

Then $Q = \{ Q_k |Q_k$ is a generator $\}$ is a division of $G_I$ by $H_1$ if
for all k, $1 < k < n$, $Q_k$ = $( \, q_{k1}, q_{k2}, \ldots, q_{kn} ) \,$ where
\[
	\begin{cases}
		g_j - H_1, \text{ if } j < k,\\
		g_j \cap H_1, \text{ if } j = k,\\
		g_j \text{ if } j > k
	\end{cases}
\]


\par 
\textbf{Definition} 3.11:
\newline

\par 
\textbf{Lemma} 3.11:
\newline

\par 
\textbf{Definition} 3.12:
\newline

\par 
\textbf{Lemma} 3.12:
\newline


\par 
\textbf{Definition} 3.13:
\newline

\par 
\textbf{Lemma} 3.13:
\newline


\par 
\textbf{Definition} 3.14:
\newline

\par 
\textbf{Lemma} 3.14:
\newline



\par 
\textbf{Definition} 3.15:
\newline

\par 
\textbf{Lemma} 3.15:
\newline


\par 
\textbf{Lemma} 3.16:
\newline

\par 
\textbf{Lemma} 3.17:
\newline

\par 
\textbf{Lemma} 3.18:
\newline

\par 
\textbf{Thereom} 3.19:
\newline


\par 
\textbf{Definition} 3.16:
\newline

\par 
\textbf{Lemma} 3.20:
\newline

\par 
\textbf{Definition} 3.17:
\newline

\par 
\textbf{Lemma} 3.21:
\newline


\par 
\textbf{Definition} 3.18:
\newline

\par 
\textbf{Lemma} 3.22:
\newline

\par 
\textbf{Thereom} 3.23:
\newline

\par 
\textbf{Definition} 3.19:
\newline

\par 
\textbf{Definition} 3.20:
\newline

\par 
\textbf{Lemma} 3.24:
\newline

TBD

\par 
\textbf{Thereom} 3.25:
\newline










\end{document}
